%!TeX program = xelatex
\documentclass{CUGReport}

% 根据个人情况修改
\headl{左侧页眉}
\headc{中间页眉}
\headr{右侧页眉}
\lessonTitle{XXX课程实习报告}
\reportTitle{XXX系统设计}
\stuname{张三}
\stuid{20300000}
\inst{机械与电子信息学院}
\major{通信工程}
\time{2024年7月1日-7月20日}
\date{\today}

\begin{document}

% =============================================
% Part 1: 封面
% =============================================
\cover
\thispagestyle{empty} % 首页不显示页码
\clearpage

% =============================================
% Part 2: 摘要
% =============================================
\begin{abstract}

实验报告的结构包括:标题页、摘要、实验内容、方法论、实验结果、讨论、参考文献及附录,其中标题页、摘要、参考文献和附录都要单独占用一页。写摘要的目的是概括报告的内容,不加评论和补充解释,篇幅以150字为宜。摘要应该简洁、具体地反映报告的实质性信息,包括:实验目的及理由;参加人和计划 ;实验方法论;最重要的发现和观点;该实验的意义和重要性。需要注意,摘要位于报告的开头,但最好在整个报告写完之后再来写摘要。实验内容写明依据何种原理,提供该研究的理论基础。研究目标是研究者希望通过实验达到的目的。实验内容这一部分需要:在理论基础上,简明地介绍你的题目,并且确定关键词;解释理论框架。值得关注的是:要保持思维的清晰性和逻辑性;没必要详细地介绍该研究的过程和细节;不建议以一种写文章的思路去写摘要。实验方法需要简单地介绍一下你所使用的研究方法。这一部分的要求可以归纳为以下几点:写作时,记得使用第三人称;一定要使用过去时态;为了让别人能够验证你的实验,需要提供该实验的细节;没必要解释为什么你选择了某种方法,告诉你做了什么即可。实验结果一般提供实验现象的描述和实验数据的处理。在撰写实验结果的过程中,同样需要使用过去时态,而且保持清晰性和简洁性。你只需要列出结果,没必要解释它们(讨论部分才解释结果)。此外,不要提供任何原始数据,但是如果有表格或图表做补充说明,那就更好了。在相关的理论知识的基础上,对所得到的实验结果进行解释和分析。讨论需要提到的问题包括:如果你的实验结果和预期的结果一致,那么它可以验证什么理论?实验结果有什么意义?说明了什么问题?写作时,需要注意以下几点:总结一下你们的发现,而且把结果与假设联系起来;把结果与介绍中的背景资料对比一下,然后提出你的结论 ;提一些建设性的建议改进你的研究 (如果有机会的话);写一下本研究的发现对人们在现实生活中的行为方式有什么意义;概括一下你的发现和讨论的要点 (用2-3句话)。


\vspace{4em}
\textbf{关键词:} \textbf{实验报告;格式;模板;\LaTeX;}
\end{abstract}
 \thispagestyle{empty} % 摘要页不显示页码
 \clearpage


% =============================================
% Part 3: 目录页
% =============================================
% 重置页码,并使用罗马数字
\pagenumbering{Roman}
\setcounter{page}{1}
\tableofcontents
\clearpage

% =============================================
% Part 4: 正文内容
% =============================================
% 重置页码,并使用阿拉伯数字
\pagenumbering{arabic}
\setcounter{page}{1}



%%可选择这里也放一个标题
%\begin{center}
%    \title{ \Huge \textbf{{标题}}}
%\end{center}

\section{模板说明}

本模板主要适用于一些课程的实验报告以及期末论文,默认页边距为2.5cm,中文宋体,英文Times New Roman,字号为12pt(小四)。

默认带有封面页以及目录页,目录部分页码和正文单独计数。

\subsection{\LaTeX{}与模板介绍}

\LaTeX{}(发音为 “Lah-tech” 或 “Lay-tech” )是由 Leslie Lamport 开发的当今世界上最流行和使用最为广泛的 \TeX{} 宏集。它构筑在 PlainTeX 的基础之上,并加进了很多的功能以使得使用者可以更为方便的利用 \TeX{} 的强大功能。

使用 \LaTeX{} 基本上不需要使用者自己设计命令和宏等,因为 \LaTeX{} 已经替你做好了。因此,即使使用者并不是很了解 \TeX{},也可以在短短的时间内生成高质量的文档。对于生成复杂的数学公式,\LaTeX{} 表现的更为出色。

\LaTeX{} 由 \LaTeX{3} 项目维护,很多使用者对 \LaTeX{} 加入了很多补充扩展,例如为 \LaTeX{} 开发宏包和样式,其中的一些已经包含在很多 \LaTeX{} 软件中,可以在CTAN上获得更多的扩展宏包。

编译方式:\verb|xelatex -> bibtex -> xelatex*2|

默认模板文件由以下四部分组成:

\begin{itemize}
    \item \texttt{main.tex} 主文件,报告内容在此输入
    \item \texttt{reference.bib} 参考文献,使用bibtex格式
    \item \texttt{CUGReport.cls} 文档格式控制,包括一些基础的设置,如页眉、标题、姓名等的格式设置,一般不需要修改
    \item \texttt{figures} 放置图片的文件夹
\end{itemize}



\subsection{模板使用方法}

使用模板时,我们只需要在\texttt{main.tex} 中\verb|\begin{document}|……
\verb|\end{document}|之间输入你的报告内容就好,这两个命令表示了文档的开始和结束。命令\verb|\section{节标题}|表示一节的开始,\verb|\subsection{小节标题}|表示小节的开始。在文档中间,就是文档的主要内容,包括节、小节、图、表、公式、代码和参考文献等各种元素,这些就是我们的主要工作,只要这些元素的描述语言正确,就可以得到格式统一、美观的文档。在使用前,可以对照\texttt{main.tex}内容和生成的\texttt{main.pdf}文件内容和格式进行学习,依葫芦画瓢,多动手练习,很快就能掌握基本的用法,并得到排版美观、格式标准的实验报告。

如果想修改文档的生成格式,可以在学习\LaTeX{}语言的基础上修改、设置模板中对应的变量参数,当然也可以去\texttt{CUGReport.cls}中修改对应的设置,制作属于你自己的模板。

推荐学习平台\url{www.texpage.com},或者\url{www.overleaf.com}。这两个平台都是在线编辑、编译平台,注册之后通过浏览器上传此模板压缩包,然后就直接可以使用,不用自己安装\LaTeX{}环境,并且还提供了公式编辑器、特殊符号输入、协作审阅等功能,在使用时可以慢慢了解。在\url{https://www.texpage.com/docs/zh/} 的文档中心提供了\LaTeX{}的基本知识以及基本功能的介绍,可以作为使用参考。



\section{一些插入功能}

\subsection{数学符号}

中文论文的数学符号遵循 GB/T 3102.11—1993《物理科学和技术中使用的数学符号》标准\footnote{原 GB 3102.11—1993,自2017年3月23日起,该标准转为推荐性标准。},并参照 ISO 80000-2:2019。英文论文的数学符号采用 \TeX{} 默认的样式。

推荐使用\href{http://mirrors.ctan.org/macros/latex/contrib/siunitx/siunitx.pdf}{siunitx} 宏包处理量和单位,以便方便地处理希腊字母以及数字与单位之间的空白。例如:
\SI{6.4e6}{m},
\SI{9}{\micro\meter},
\SI{30}{kg.m.s^{-1}},
\SI{20}{\degreeCelsius}。

 \autoref{tab:number} 展示了一些数字和单位的正确写法以及常见的错误写法。

\vspace{0.1em}
\begin{table}[H]
    \centering
    \caption{数字与单位示范}
    \label{tab:number}
    \renewcommand\arraystretch{0.85} % 定义表格行距
    \setlength{\tabcolsep}{15pt} % 定义列间宽度
    \begin{tabular}{@{}cc@{}}
        \toprule[1.5pt]
        \textbf{正确示例} & \textbf{错误示例} \\ 
        \midrule[0.8pt]
        \num{12345,67890} & 12345.67890 \\
        \num{.3e45} & 0.3 $\times$ 10\textsuperscript{45} \\
        \si{\kilo\gram\metre\per\square\second} & kg m s\textsuperscript{-2} \\
        \si{\square\volt\cubic\lumen\per\farad} & $V^{2}lm^{3}F^{-1}$ \\
        \SI[mode=text]{1.23}{J.mol^{-1}.K^{-1}} & 1.23J mol\textsuperscript{-1}K\textsuperscript{-1} \\
        \SI[per-mode=symbol]{1.99}[\$]{\per\kilogram} & \$ 1.99/kg \\
        \SI[per-mode=fraction]{1,345}{\coulomb\per\mole} & 1.345$\frac{C}{mol}$ \\ 
        \bottomrule[1.5pt]
    \end{tabular}
\end{table}
\vspace{-0.5em}  % 减少表格与正文间的间距

\subsection{插入公式}
插入公式分为两种,一种是行内公式,一种是行间编号(无编号)公式。行内公式插入方法比较简单,在需要输入公式的地方,用如下代码就可以插入简单的行内公式,其中“\$ ”是行内公式的提示符,行内公式必须以此符号开始和结束。
\begin{verbatim}
    $v-\varepsilon+\phi=2$
\end{verbatim}
行内公式$v-\varepsilon+\phi=2$。

插入行间公式如\autoref{eq:Eq1}:

\begin{equation}
     x_t = a_0 + a_1x_{t-1} + a_2x_{t-2} + \epsilon_t
    \label{eq:Eq1}
\end{equation}

可以看到,此时公式右侧会自动编号。插入行间编号公式的代码如下。
\begin{verbatim}
    \begin{equation}
         x_t = a_0 + a_1x_{t-1} + a_2x_{t-2} + \epsilon_t
        \label{eq:Eq1}
    \end{equation}
\end{verbatim}

与此相似,插入行间无编号公式的代码如下。
\begin{verbatim}
    \begin{equation*}
         x_t = a_0 + a_1x_{t-1} + a_2x_{t-2} + \epsilon_t
        \label{eq:Eq2}
    \end{equation*}
\end{verbatim}
得到结果如下。
\begin{equation*}
     x_t = a_0 + a_1x_{t-1} + a_2x_{t-2} + \epsilon_t
     \label{eq:Eq2}
\end{equation*}

对于多行公式,如果不需要按等号对齐, 可以使用“\{gather\} ” 环境,如下:
\abovedisplayshortskip=3pt
\belowdisplayshortskip=4pt
\abovedisplayskip=3pt
\belowdisplayskip=4pt
\begin{gather*}%不会产生编号
max \ \ Cov(t_1,u_1) = Cov(Xw_1,Yv_1)\\
s.t.
\begin{cases}
  w_1^Tw_1 = \left \| w_1 \right \|^2 = 1  \\
  v_1^Tv_1 = \left \| v_1 \right \|^2 = 1  
\end{cases}
\end{gather*}

而如果需要多行公式尽可能在等号处对齐,可以使用 “\{align\}” 环境。
\abovedisplayshortskip=4pt
\belowdisplayshortskip=4pt
\abovedisplayskip=4pt
\belowdisplayskip=4pt
\begin{align}
Y &= t_1r_1^T + t_2r_2^T + \cdots + t_mr_m^T + Y_m \\
  &= (Xw_1^*)r_1^T + (Xw_2^*)r_2^T + \cdots + (Xw_m^*)r_m^T + Y_m \\
  &= X ({\textstyle \sum_{i=1}^{m}} w_ir_i^T) + Y_m
\end{align}

如果需要多个公式组在一起共用一个编号, 编号位于公式的居中位置,推荐使用 “\{aligned\}” 环境。使用效果如公式 \autoref{eq:example2} 所示。
\abovedisplayshortskip=4pt
\belowdisplayshortskip=4pt
\abovedisplayskip=4pt
\belowdisplayskip=4pt
\begin{equation}
\label{eq:example2}
\left\{
    \begin{aligned}
     & X = t_1p_1^T + t_2p_2^T + \cdots + t_mp_m^T \\
     & Y = t_1r_1^T + t_2r_2^T + \cdots + t_mr_m^T + Y_m
    \end{aligned}
\right.
\end{equation}

\subsection{插入图片}
\subsubsection{单个图片}
图片通常在 figure 环境中使用 includegraphics 插入,如图 \ref{fig:CUG} 的源代码。建议使用矢量图片(PDF)。照片建议使用 JPG 格式。其他的栅格图建议使用无损的 PNG 格式。图片可以通过 width 参数来设置宽度,设置宽度后长度会等比例放缩。一般 width 参数会搭配 \verb|\linewidth| 使用,以实现按照当前页面宽度进行等比例放缩。另外可以通过\verb|\vspace|,来调整图片与上下文之间的间距。
CUG校徽如\autoref{fig:CUG}所示,注意这里使用了\verb|\autoref{}|命令,也就是会自动生成“图”“式”等前缀,无需手动输入。
插入\autoref{fig:CUG}的代码:

\begin{verbatim}
    \begin{figure}[!hbp]    %[]中为图片位置放置方式及顺序
        \centering                %居中 
        %引用图片
        \includegraphics[width =.4\textwidth]
         {figures/xinxiaohui-1952(2).png}   
        \caption{中国地质大学(武汉)}    %图片名称说明文字
        \label{fig:CUG}        %图片标签
    \end{figure}                %结束图片环境
\end{verbatim}


\begin{figure}[htbp]
    \centering
    \includegraphics[width =.4\textwidth]{xinxiaohui-1952.png}
    \caption{中国地质大学(武汉)}
    \label{fig:CUG}
\end{figure}

\subsubsection{多个图片}
该模板使用 \verb|subcaption| 宏包来处理分图,呈现了如 \autoref{fig:CUG-01} 、 \autoref{fig:CUG-02} 和 \autoref{fig:CUG-03}所示的效果。

在 \autoref{fig:CUG-01} 中,通过 \verb|\subcaptionbox| 命令展示了两个独立的子图( \autoref{fig:subfig-a-1} 和 \autoref{fig:subfig-b-1}),分别使用不同的标题,并被包含在总图标题“多个独立标题分图示例 01”中。

\begin{figure}[H]
    \centering
    \subcaptionbox{sub caption A \label{fig:subfig-a-1}}{\includegraphics[width=0.35\linewidth]{figures/xinxiaohui-1952.png}}
    \quad        %两图之间的间距,当前字体下一个“M”的宽度
    \subcaptionbox{sub caption B \label{fig:subfig-b-1}}{\includegraphics[width=0.35\linewidth]{figures/xinxiaohui-1952(2).png}}
    \caption{多个独立标题分图示例 01}
    \label{fig:CUG-01}
\end{figure}
\vspace{-0.7em}  % 减少图片与正文间的间距

\begin{figure}[H]
    \centering
    \subcaptionbox{分图 A1\label{fig:subfig-a-2-01}}{\includegraphics[width=0.17\linewidth]{figures/xinxiaohui-1952.png}}
    \subcaptionbox{分图 B1\label{fig:subfig-b-2-01}}{\includegraphics[width=0.17\linewidth]{figures/xinxiaohui-1952(2).png}}
    \subcaptionbox{分图 A2\label{fig:subfig-a-2-02}}{\includegraphics[width=0.17\linewidth]{figures/xinxiaohui-1952.png}}
    \subcaptionbox{分图 B2\label{fig:subfig-b-2-02}}{\includegraphics[width=0.17\linewidth]{figures/xinxiaohui-1952(2).png}}
    \caption{多个独立标题分图示例 02}
    \label{fig:CUG-02}
\end{figure}
\vspace{-0.7em}  % 减少图片与正文间的间距

多个分图可以以多行的形式展示,使用的效果如图 \autoref{fig:CUG-03} 所示。

\begin{figure}[H]
    \centering
    \subcaptionbox{分图 A1\label{fig:subfig-a1-3}}{\includegraphics[width=0.25\linewidth]{figures/xinxiaohui-1952.png}}
    \subcaptionbox{分图 A2\label{fig:subfig-a2-3}}{\includegraphics[width=0.25\linewidth]{figures/xinxiaohui-1952(2).png}}
    \\
    \subcaptionbox{分图 B1\label{fig:subfig-b1-3}}{\includegraphics[width=0.25\linewidth]{figures/xinxiaohui-1952(2).png}}
    \subcaptionbox{分图 B2\label{fig:subfig-b2-3}}{\includegraphics[width=0.25\linewidth]{figures/xinxiaohui-1952.png}}
    \caption{多个独立标题分图示例 03}
    \label{fig:CUG-03}
\end{figure}
\vspace{-0.7em}  % 减少图片与正文间的间距

minipage 也可以实现排版并排插图, minipage 可以划分出虚拟的区块,每个区块中可以进行独立排版,使用的效果如图 \autoref{fig:minipage-1} 和  \autoref{fig:minipage-2} 所示。
\begin{figure}[H]
    \centering
    \begin{minipage}[Ht]{0.48\linewidth} % 第一个minipage
        \centering
        \includegraphics[width=0.48\linewidth]{figures/xinxiaohui-1952.png}
        \caption{图 A}
        \label{fig:minipage-1}
    \end{minipage}
    \begin{minipage}[Ht]{0.48\linewidth} % 第二个minipage
        \centering
        \includegraphics[width=0.48\linewidth]{figures/xinxiaohui-1952(2).png}
        \caption{图 B}
        \label{fig:minipage-2}
    \end{minipage}
\end{figure}
\vspace{-0.7em}    %减少图片与正文间的间距

以上就是插入图片的几种基本形式,可以根据自己的需要选择合适的图片排版形式进行图片的展示。

\subsection{插入文本框}
本模板定义了一个圆角灰底的文本框,使用简化命令\verb|\tbox{}|即可,如果你不喜欢,可以前往 \texttt{CUGReport.sty}对其进行修改。

\tbox{
    这是一个圆角灰底的文本框
}

\subsection{插入表格}

\subsubsection{普通表格}
在 \LaTeX{} 中,表格的编辑相对较为复杂,推荐使用 Table Generator\footnote{Table Generator 网址:\url{https://www.tablesgenerator.com/}} 来生成表格。为使表格结构更加简洁通用,通常需要使用三线表。三线表是传统网格线表经过简化改造而来的,取消了斜线、竖线和横向分割线, \autoref{tab:three-line} 是一个三线表的示例。

\begin{table}[H] % 表格位置固定
    \centering % 表格整体居中
    \caption{三线表示例} % 表格表题
    \label{tab:three-line} % 表格标签
    \renewcommand\arraystretch{0.85} % 定义表格行距
    \setlength{\tabcolsep}{12pt} % 定义列间宽度
    \begin{tabular}{ccc} % 表格列样式定义:3列,每列居中
        \toprule[1.5pt] % 顶线
        \textbf{列名1} & \textbf{列名2} & \textbf{列名3} \\ % 表头
        \midrule[0.8pt] % 栏目线
            Wuhan & Wuhan & Wuhan \\ % 表体
            Yantai & Yantai & Yantai \\ % 表体
            Yantai & Yantai & Yantai \\ % 表体
        \bottomrule[1.5pt] % 底线
    \end{tabular}
\end{table}
\vspace{-0.5em}  % 减少表格与正文间的间距

比较简单的表格文件所示表格如\autoref{tab:doctab1}所示。

\begin{table}[!htbp]
    \centering
    \caption{本模板文件组成}    %表名
    \begin{tabular}{c|c}        %“{c|c}”中'|'表示两列之间有分割线,c表示居中
        \hline                           %最上横线
        文件名                  & 说明         \\%列名
        \hline                            %中间横线
        \texttt{main.tex}       & 主文件       \\    %第一行
        \texttt{reference.bib}  & 参考文献     \\%第二行
        \texttt{SYSUReport.sty} & 文档格式控制 \\    %第三行
        \texttt{figures}        & 图片文件夹   \\    %第四行
        \hline                        %最下横线
    \end{tabular}
    \label{tab:doctab1}
\end{table}

\subsubsection{长表格}

如某个表需要转页接排,可以使用 longtable 宏包,需要在随后的各页上重复表的编号。
编号后跟表题(可省略)和“(续)”,置于表上方。续表均应重复表头。如表 \autoref{tab:longTable} 所示。不过当一个张表内容过多时,建议将该表置于附录中。

\renewcommand\arraystretch{0.85} % 定义表格行距,注意命令的位置与普通表格不同
\begin{longtable}{cccccccc}
    \caption{跨页长表格} \\ % 换页前标题
    
    \toprule[1.5pt]
        \textbf{烟台} & \textbf{长沙} & \textbf{南昌} & \textbf{婺源} & \textbf{天津} & \textbf{上海} & \textbf{北京} & \textbf{青岛} \\ % 换页前表头
    \midrule[0.75pt]
    \endfirsthead
    
    \caption[]{跨页长表格(续)} \\ % 换页后标题
    \toprule[1.5pt]
        \textbf{烟台} & \textbf{长沙} & \textbf{南昌} & \textbf{婺源} & \textbf{天津} & \textbf{上海} & \textbf{北京} & \textbf{青岛} \\  % 换页后表头
    \midrule[0.75pt]
    \endhead
        \bottomrule[1.5pt]
        % \multicolumn{7}{r}{\textit{\zihao{5} \songti 接下页}} \\ 
    \endfoot 
    \bottomrule[1.5pt]
    \endlastfoot
    \label{tab:longTable}
        Row 01 & 01-01 & 01-02 & 01-03 & 01-04 & 01-05 & 01-06 & 01-07 \\
        Row 02 & 02-01 & 02-02 & 02-03 & 02-04 & 02-05 & 02-06 & 02-07 \\
        Row 03 & 03-01 & 03-02 & 03-03 & 03-04 & 03-05 & 03-06 & 03-07 \\
        Row 04 & 04-01 & 04-02 & 04-03 & 04-04 & 04-05 & 04-06 & 04-07 \\
        Row 05 & 05-01 & 05-02 & 05-03 & 05-04 & 05-05 & 05-06 & 05-07 \\
        Row 06 & 06-01 & 06-02 & 06-03 & 06-04 & 06-05 & 06-06 & 06-07 \\
        Row 07 & 07-01 & 07-02 & 07-03 & 07-04 & 07-05 & 07-06 & 07-07 \\
        Row 08 & 08-01 & 08-02 & 08-03 & 08-04 & 08-05 & 08-06 & 08-07 \\
        Row 09 & 09-01 & 09-02 & 09-03 & 09-04 & 09-05 & 09-06 & 09-07 \\
        Row 10 & 10-01 & 10-02 & 10-03 & 10-04 & 10-05 & 10-06 & 10-07 \\
        Row 11 & 11-01 & 11-02 & 11-03 & 11-04 & 11-05 & 11-06 & 11-07 \\
\end{longtable}
\vspace{-0.5em}  % 减少表格与正文间的间距

\subsection{插入代码}
\subsubsection{代码块}
代码环境可以使用 \verb|{lstlisting}| 宏包。宏包可以自定义代码中关键字的高亮、代码块的边框、行号的样式等。代码 \ref{code:Python} 是一段 Pyhton 代码示例。具体使用请查看源文件。

\vspace{-0.5em}
\begin{lstlisting}[label=code:Python, language=Python, caption=Python 代码示例]
sns.set(style='whitegrid', font_scale=1.3, rc={'figure.figsize': (20, 15), 'axes.edgecolor': '0.5'})
# Create a canvas with 15 subplots
fig, axes = plt.subplots(nrows=3, ncols=5, figsize=(26, 15), gridspec_kw={'wspace': 0.4, 'hspace': 0.4})
# colors
color = '#3498db'
# plot histgram
for i, ax in enumerate(axes.flatten()):
    if i < len(data.columns): 
        ax.hist(data.iloc[:, i], bins=20, color=color, edgecolor='black')
        ax.set_title(data.columns[i], fontsize=18)  
for i in range(len(data.columns), 3 * 5):
    axes.flatten()[i].axis('off')
\end{lstlisting}
\vspace{0.1em}

\subsubsection{插入伪代码}
伪代码(算法)环境可以使用\verb|{algorithms}|宏包。算法\ref{alg:decision_tree}为演示的示例。
\begin{algorithm}
	\caption{Decision Tree Algorithm}
	\label{alg:decision_tree}
	\begin{algorithmic}[1]
		\STATE \textbf{Input:} Training dataset $D$, set of features $F$, current node $N$
		\STATE \textbf{Output:} Decision tree $DT$
		
		\IF{Stopping criteria are met for node $N$}
			\STATE Create a leaf node with the majority class in $D$
		\ELSE
			\STATE Select the best feature $f$ to split on
			\STATE Split $D$ into subsets $D_1$ and $D_2$ based on the value of feature $f$
			\STATE Create a decision node for $N$ with the split criterion $f$
			\STATE Recursively build the left subtree: $DT.left \leftarrow \text{BuildDecisionTree}(D_1, F, N_{\text{left}})$
			\STATE Recursively build the right subtree: $DT.right \leftarrow \text{BuildDecisionTree}(D_2, F, N_{\text{right}})$
		\ENDIF
		
		\RETURN $DT$
	\end{algorithmic}  
\end{algorithm}
\vspace{-0.5em}  % 减少表格与正文间的间距

\section{定理环境}
\begin{Theorem}
\end{Theorem}

\begin{Lemma}
\end{Lemma}

\begin{Corollary}
\end{Corollary}

\begin{Proposition}
\end{Proposition}

\begin{Definition}
\end{Definition}

\begin{Example}
\end{Example}

\begin{proof}
\end{proof}

\subsection{插入参考文献}

模板使用 \hologo{BibTeX} 处理引用参考文献。使用时,将bibtex格式参考文献信息保存在reference.bib文件中,然后就可以在正文中直接引用,同时也会在报告的参考文献部分生成标准格式的参考文献列表,非常方便。本节主要介绍 \hologo{BibTeX} 配合 \verb|natbib| 宏包的主要使用方法\cite{mao2021correlation}。

直接使用\verb|\cite{}|即可。

例如:此处引用了文献\verb|\cite{noauthor_undated-qy}|。结果为:此处引用了文献~\cite{noauthor_undated-qy}。

\par 引用过的文献会自动出现在报告后面的参考文献部分中。

\subsubsection{顺序编码方式}

在顺序编码制下,默认的 \verb|\cite{}| 命令同 \verb|\citep{}| 一样,序号置于方括号中,引文页码会放在括号外。同一处引用的连续序号会自动用短横线连接。

\begin{table}[H]
  \centering
  \caption{顺序编码制中的对应关系}
      \begin{tabular}{l@{\quad$\Rightarrow$\quad}l}        %在表格中插入字符用‘@’符号
      \verb|\cite{mao2021correlation}| & \cite{mao2021correlation}   \\
      \verb|\citet{mao2021correlation}| & \citet{mao2021correlation} \\
      \verb|\citep{mao2021correlation}| & \citep{mao2021correlation} \\
      \verb|\cite[42]{mao2021correlation}| & \cite[42]{mao2021correlation}  \\
      \verb|\cite{mao2021correlation,tang2023lrbmat}| & \cite{mao2021correlation,noauthor_undated-qy} \\
      \end{tabular}
\end{table}
\vspace{-0.5em}  % 减少表格与正文间的间距

也可以通过 \verb|\setcitestyle{numbers}| 设置引用样式为取消上标,将数字序号作为文字的一部分。

\setcitestyle{numbers} % 修改引用样式为取消上标格式

\begin{table}[H]
  \centering
  \caption{取消上标格式的顺序编码制中的对应关系}
      \begin{tabular}{l@{\quad$\Rightarrow$\quad}l}
      \verb|\cite{mao2021correlation}| & \cite{mao2021correlation} \\
      \verb|\citet{mao2021correlation}| & \citet{mao2021correlation} \\
      \verb|\citep{mao2021correlation}| & \citep{mao2021correlation} \\
      \verb|\cite[42]{mao2021correlation}| & \cite[42]{mao2021correlation}  \\
      \verb|\cite{mao2021correlation,tang2023lrbmat}| & \cite{mao2021correlation,noauthor_undated-qy} \\
      \end{tabular}
\end{table}
\vspace{-0.7em}  % 减少表格与正文间的间距

\subsubsection{著者-出版年方式}

可以通过 \verb|\setcitestyle{authoryear}| 设置属性引用样式为著者-出版年制,其中 \verb|\cite{}| 
跟 \verb|\citet{}| 效果一样。

\setcitestyle{authoryear} % 修改引用样式为著者-出版年制

\begin{table}[H]
  \centering
  \caption{著者-出版年制中的对应关系}
      \begin{tabular}{l@{\quad$\Rightarrow$\quad}l}
      \verb|\cite{mao2021correlation}| & \cite{mao2021correlation} \\
      \verb|\citet{张天骐2024基于}| & \citet{张天骐2024基于} \\
      \verb|\citep{吴修振2024一种基于声波信号的室内定位方法}| & \citep{吴修振2024一种基于声波信号的室内定位方法} \\
      \verb|\cite[42]{mao2021correlation}| & \cite[42]{mao2021correlation}  \\
      \end{tabular}
\end{table}
\vspace{-0.5em}  % 减少表格与正文间的间距

\setcitestyle{super} % 修改引用样式为默认

\section{写在最后}

\subsection{说明}

本项目在\href{https://github.com/NorthSecond/SYSU_Latex_Template}{中山大学模板}的基础上进行修改,并参考了\href{https://github.com/zcyeee/HNU_LaTeX_Template}{湖南大学课程论文模板}中部分内容,感谢原作者们的辛勤付出。本模板还存在较多问题,欢迎有兴趣的同学修改完善。

\subsection{发布地址}

\begin{itemize}
    \item Github: \url{https://github.com/zcyeee/HNU_LaTeX_Template}
    \item Github: \url{https://github.com/Yangmomo1978/CUG_Report_Template}
    \item Github: \url{https://github.com/NorthSecond/SYSU_Latex_Template}
    \item Overleaf: \url{https://www.overleaf.com/latex/templates/sysu-latex-template/dxwrhzbydxyq}
    \item TexPage: \\
          \url{https://www.texpage.com/template/21db014e-5065-448c-a6f2-545b983aee2d}
\end{itemize}

% =============================================
% Part 5: 参考文献
% =============================================
%在reference.bib文件中填写参考文献,此处自动生成

\newpage
%bibliographystyle{ieeetran}    %在.cls文件中已经定义了引用格式
\bibliography{reference.bib}

\newpage
\StartAppendix % 启用附录

\section{附录}

附录


\StartAcknowledgements % 启用致谢

为了完成此模板,参考了众多学校的模板和设计代码,再次感谢无私奉献的开源者们!

\end{document}
